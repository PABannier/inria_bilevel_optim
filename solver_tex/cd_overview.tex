\documentclass[a4paper,10pt]{article}
\usepackage[margin=2.5cm]{geometry}

\usepackage{amssymb,amsmath,amsthm}
\usepackage{color}
\usepackage{enumitem}
\usepackage{dsfont}
\usepackage{bm}




\newtheorem{theorem}{Theorem}
\newtheorem{proposition}{Proposition}
\newtheorem{cor}{Corollary}
\theoremstyle{definition}
\newtheorem{definition}{Definition}
\newtheorem{remark}{Remark}
\newtheorem{example}{Example}
\newtheorem{claim}{Claim}
\newtheorem{lemma}{Lemma}

%%%%%%%%%%%%%%%%%%%%%%%%%%%%%%%%%%%%%%%%%%%%%%%%%%%%%%%%%%%%%%%%%%%%%%%%%%%%%%%
% Cross-referencing
%%%%%%%%%%%%%%%%%%%%%%%%%%%%%%%%%%%%%%%%%%%%%%%%%%%%%%%%%%%%%%%%%%%%%%%%%%%%%%%

\usepackage{hyperref}

%%%%%%%%%%%%%%%%%%%%%%%%%%%%%%%%%%%%%%%%%%%%%%%%%%%%%%%%%%%%%%%%%%%%%%%%%%%%%%%
% Algorithms
%%%%%%%%%%%%%%%%%%%%%%%%%%%%%%%%%%%%%%%%%%%%%%%%%%%%%%%%%%%%%%%%%%%%%%%%%%%%%%%

\usepackage{algorithm}
\usepackage{algorithmic}
\usepackage[titlenumbered,ruled,noend,algo2e]{algorithm2e}
\newcommand\mycommfont[1]{\footnotesize\ttfamily\textcolor{blue}{#1}}
\SetCommentSty{mycommfont}
\SetEndCharOfAlgoLine{}


%%%%%%%%%%%%%%%%%%%%%%%%%%%%%%%%%%%%%%%%%%%%%%%%%%%%%%%%%%%%%%%%%%%%%%%%%%%%%%%
% Code
%%%%%%%%%%%%%%%%%%%%%%%%%%%%%%%%%%%%%%%%%%%%%%%%%%%%%%%%%%%%%%%%%%%%%%%%%%%%%%%


\usepackage{fancyvrb}                  % for fancy verbatim
\usepackage{textcomp}
\usepackage[space=true]{accsupp}
% requires the latest version of package accsupp
\newcommand{\copyablespace}{
    \BeginAccSupp{method=hex,unicode,ActualText=00A0}
\ %
    \EndAccSupp{}
}
\usepackage[procnames]{listings}
% \usepackage{setspace} % need for \setstretch{1}
\lstset{%
language   = python,%
 % basicstyle = \ttfamily\setstretch{1},%
basicstyle = \ttfamily,%
columns    = flexible,%
keywordstyle=\color{javared},
firstnumber=100,
frame=shadowbox,
showstringspaces=false,
morekeywords={import,from,class,def,for,while,if,is,in,elif,
else,not,and,or,print,break,continue,return,True,False,None,access,
as,del,except,exec,finally,global,import,lambda,pass,print,raise,try,assert,!=},
keywordstyle={\color{javared}\bfseries},
commentstyle=\color{javagreen}, %vsomb_col white comments
morecomment=[s][\color{javagreen}]{"""}{"""},
upquote=true,
%% style for number
numbers=none,
resetmargins=true,
xleftmargin=10pt,
linewidth= \linewidth,
numberstyle=\tiny,
stepnumber=1,
numbersep=8pt, %
frame=shadowbox,
rulesepcolor=\color{black},
procnamekeys={def,class},
procnamestyle=\color{oneblue}\textbf,
literate={á}{{\'a}}1
{à}{{\`a }}1
{ã}{{\~a}}1
{é}{{\'e}}1
{ê}{{\^e}}1
{è}{{\`e}}1
{í}{{\'i}}1
{î}{{\^i}}1
{ó}{{\'o}}1
{õ}{{\~o}}1
{ô}{{\^o}}1
{ú}{{\'u}}1
{ü}{{\"u}}1
{ç}{{\c{c}}}1
}


\usepackage{times} % use Times

\usepackage{shortcuts_js} % possibly adapted from https://github.com/josephsalmon/OrganizationFiles/sty/shortcuts_js.sty

%%%%%%%%%%%%%%%%%%%%%%%%%%%%%%%%%%%%%%%%%%%%%%%%%%%%%%%%%%%%%%%%%%%%%%%%%%%%%%%
% IMAGES
%%%%%%%%%%%%%%%%%%%%%%%%%%%%%%%%%%%%%%%%%%%%%%%%%%%%%%%%%%%%%%%%%%%%%%%%%%%%%%%

% Use prebuiltimages/ for images extracted from code (e.g. python)
% or to share images built from a software not available by the whole team (say matlab .fig, or inskcape .svg).
% .svg files should be stored in dir srcimages/ and built from moosetex if needed:
% https://www.charles-deledalle.fr/pages/moosetex.php
% NEVER (GIT) versions files in images/ : only prebuiltimages/ & srcimages/ !

\usepackage{graphicx} % For figures
\graphicspath{{images/}, {prebuiltimages/}}
\usepackage{subcaption}


%%%%%%%%%%%%%%%%%%%%%%%%%%%%%%%%%%%%%%%%%%%%%%%%%%%%%%%%%%%%%%%%%%%%%%%%%%%%%%%
% For citations
%%%%%%%%%%%%%%%%%%%%%%%%%%%%%%%%%%%%%%%%%%%%%%%%%%%%%%%%%%%%%%%%%%%%%%%%%%%%%%%

\usepackage[authoryear]{natbib}
\usepackage{cleveref} % mandatory for no pbs with hyperlinks theorem etc\dots
\crefformat{equation}{Eq.~(#2#1#3)} % format for equations
\Crefformat{equation}{Equation~(#2#1#3)} % format for equations


%%%%%%%%%%%%%%%%%%%%%%%%%%%%%%%%%%%%%%%%%%%%%%%%%%%%%%%%%%%%%%%%%%%%%%%%%%%%%%%
% Header and document start
%%%%%%%%%%%%%%%%%%%%%%%%%%%%%%%%%%%%%%%%%%%%%%%%%%%%%%%%%%%%%%%%%%%%%%%%%%%%%%%


\author{Pierre-Antoine Bannier}
\title{How to solve the LASSO?}

\begin{document}

\maketitle

\vskip 0.3in

Now that we have studied LASSO and its variations as well as a brief overview of 
duality, we can now focus on the optimization algorithms that solve the LASSO problems.
In particular, we will primarily focus on coordinate descent and its variants applied 
to solve LASSO and Multi-task LASSO. 



\section*{Coordinate descent}
\label{section_1}

\subsection*{Intro to CD}

The intuition of coordinate descent is to iterate over the $p$ features and optimize the
objective function one coordinate (feature) at a time. CD implies that minimizing a function
over all its coordinates one by one leads to the convergence to a global minimum.
\\
\\
\underline{Proof}: Consider $f : \bbR^p \rightarrow \bbR$ a smooth and convex function.
Let $e_i$ be the $i$-th vector in the canonical basis of $\bbR^p$. 
\\
\\
First, 
\begin{equation*}
    \forall d \in \bbR, f(x + de_i) \geq f(x) 
    \Rightarrow 
    \frac{\partial f}{\partial x^{(i)}}(x) = 0
    \enspace .
\end{equation*}
%
Therefore,
\begin{equation*}
    \nabla f(x) = \left(
        \frac{\partial f}{\partial x^{(1)}}(x),
        \dots,
        \frac{\partial f}{\partial x^{(p)}}(x)
    \right) = 0
    \enspace .
\end{equation*}
%
which shows that for a convex and smooth function, we reach a global minimum.

\subsection*{Exact coordinate descent}

Exact coordinate descent consists in cycling through the coordindates in any order
(using any permutation of $1, \dots, p$) and solving a lot of 1-dimensional optimization
problems.
\\
\\
For a quick example, let's consider a simple 2-dimensional parabola that we want to minimize.
At the first iteration, we optimize the first dimension. At the second iteration, the second dimension.
At the third iteration, the first dimension. At the fourth, the second dimension... until convergence.

\subsection*{Coordinate gradient descent}

Solving 1-dimensional problems can still be expensive. Therefore, we'd rather rely on an approximation of
a step since we have plenty of iterations left. Therefore, we use gradient descent to make a step in a single
direction. More formally, let $\gamma_1, \dots, \gamma_p > 0$ be $p$ gradient descent steps. The coordinate
gradient descent writes:

\begin{align*}
    &\text{Choose} \enspace i_{k+1} \in \{1, \dots, n\} \\
    &\begin{cases}
        x_{k+1}^{(i)} = x_{k}^{(i)} - \gamma_i \nabla_i f(x_k), \quad &\text{if} \enspace i = i_{k+1} \\
        x_{k+1}^{(i)} = x_{k}^{(i)}, \quad &\text{if} \enspace i \neq i_{k+1} 
    \end{cases}
\end{align*}
%
\cite{Beck_Tetruashvili13} have found a convergence bound for functions with Lipschitz continuous
gradients. \cite{Nesterov12} proved a higher convergence speed when cycling randomly and independently
through the coordinates.

\section*{Proximal gradient descent}

\subsection*{What is a proximal operator?}

In convex optimization, objective functions might not always be smooth (differentiable). Let's consider
an objective function that is the sum of a smooth data-fitting term and a non-smooth regularizer term.

\begin{equation*}
    F(w) = L(w) + \lambda R(w)
\end{equation*}
%
where $L$ is a convex smooth function and $R$ is a simple non-smooth function. We want to
minimize such a function. Using standard gradient descent is not possible since we can't compute the
gradient of $F$ due to the non-differentiability of $R$. To circumvent this issue, we will derive an
optimality condition using subgradients. Let $w^*$ be a minimizer of $F$. Then, the Fermat's condition
writes:

\begin{equation*}
    0 \in \partial 
    \left(
        L(w^*) + \lambda R(w^*)
    \right) 
    = \nabla L(w^*) + \lambda \partial R(w^*)
    \enspace .
\end{equation*}
%
\begin{equation*}
    - \nabla L(w^*) \in \lambda \partial R(w^*)
\end{equation*}
%
We now have an optimality condition, but we still lack a procedure to minimize $L$. 
In order to have one, let's rely on the majorization-minimization paradigm. Simply put, 
we majorize a function and then we minimize that upper-bound. Using the $\mathcal{L}$-smoothness
of $L$, it follows that:

\begin{equation*}
    \forall w, y \in \bbR^p,
    \quad
    L(w) \leq L(y) + 
    \langle
        \nabla L(y),
        w-y
    \rangle +
    \frac{\mathcal{L}}{2} 
    \norm{w - y}^2
    \enspace .
\end{equation*}
%
Note that minimizing this upper bound yields the update rule of gradient descent. Now let's add
the non-smooth regularizer term:

\begin{equation*}
    \forall w, y \in \bbR^p,
    \quad
    L(w) + \lambda R(w) \leq L(y) + 
    \langle
        \nabla L(y),
        w-y
    \rangle +
    \frac{\mathcal{L}}{2} 
    \norm{w - y}^2 
    + \lambda R(w) 
    \enspace .
\end{equation*}
%
Now, we want to minimize the upper bound with respect to $w$:

\begin{align*}
    &\argmin_{w} L(y) + 
    \langle
        \nabla L(y),
        w-y
    \rangle +
    \frac{\mathcal{L}}{2} 
    \norm{w - y}^2 
    + \lambda R(w) \\
    %
    &= \argmin_{w}
    \langle
        \nabla L(y),
        w-y
    \rangle +
    \frac{\mathcal{L}}{2} 
    \norm{w - y}^2 
    + \lambda R(w) \\
    %
    &= \argmin_{w}
    \frac{1}{2}
    \norm{
        w - \left(
            y - \frac{1}{\mathcal{L}}\nabla L(y)
        \right)
    }^2_2 
    + \frac{\lambda}{\mathcal{L}}R(w) \\
    %
    &\triangleq \text{prox}_{\frac{\lambda}{\mathcal{L}}R}
    (y - \frac{1}{\mathcal{L}}\nabla L(y))
\end{align*}
%
where $\text{prox}$ is the proximal operator. From this equality, we can derive an update
rule for an optimization algorithm.

\begin{equation*}
    w^{t+1} = \text{prox}_{\frac{\lambda}{\mathcal{L}}R}
    (w^t - \frac{1}{\mathcal{L}}\nabla L(w^{t}))
    \enspace .
\end{equation*}
%
We can view $w^*$ as a fixed point of the proximal operator. Indeed, using Fermat's condition,

\begin{align*}
    -\nabla L(w^*) \in \lambda \partial R(w^*)
    & \iff w^* + \frac{1}{\mathcal{L}} \nabla L(w^*) \in w^* - (\lambda\gamma) \partial R(w^*) \\
    & \iff w^* \in (w^* - \frac{1}{\mathcal{L}} \nabla L(w^*)) - \frac{\lambda}{\mathcal{L}} \partial R(w^*) \\
    & \iff w^* = \text{prox}_{\frac{\lambda}{\mathcal{L}} R}(w^* - \frac{1}{\mathcal{L}} \nabla L(w^*))
\end{align*}

\subsection*{ISTA: Iterative Shrinkage Thresholding Algorithm}
\label{ista}

Let's consider the LASSO problem:

\begin{equation*}
    \min_{\beta \in \bbR^p} \frac{1}{2}
    \norm{y - X\beta}_2^2
    + \lambda \norm{\beta}_1
    \enspace .
\end{equation*}
%
We see that the objective function is the sum of a smooth data fitting term and
a non-smooth regularizer. We will derive the proximal operator of the LASSO problem 
in order to obtain an update rule for ISTA.
\\
\\
First, let's derive the proximal operator for $R(x)=\lambda\abs{x}$. $\text{prox}_R$ is the
minimizer of the function

\begin{equation*}
    h(u) = \begin{cases}
        h_1(u) \triangleq \lambda u + \frac{1}{2}(u-x)^2, \quad &u > 0, \\
        h_2(u) \triangleq -\lambda u + \frac{1}{2}(u-x)^2, \quad &u \leq 0.
    \end{cases}
\end{equation*}
%
If the minimizer is attained at $u > 0$ then $0 = h_1'(u)=\lambda + u - x$, meaning that
$u = x - \lambda$. Therefore, if $x > \lambda$, then $\text{prox}_R(x)=x-\lambda$. The same 
argument shows that if $x < -\lambda$, then $\text{prox}_R(x)=x+\lambda$. If $\abs{x} \leq \lambda$,
then $\text{prox}_R(x)$ must be the only point of nondifferentiability of $h$, namely, 0.
Therefore, we can define the soft-thresholding function as:

\begin{equation*}
    \mathcal{T}_{\lambda}(x) = \left[\abs{x} - \lambda\right]_+ \text{sgn}(x)
    \begin{cases}
        x - \lambda,  \quad &x \geq \lambda \\
        0, \quad & \abs{x} < \lambda \\
        x + \lambda, \quad &x \leq -\lambda
    \end{cases}
\end{equation*}
%
For a vector, we can generalize the notation as follows:

\begin{equation*}
    \text{prox}_R(\mathbf{x}) = (\mathcal{T}_{\lambda}(x_j))_{j=1}^n
    = \left[\abs{\mathbf{x}} - \lambda \mathbf{1}\right]_+ \odot \text{sgn}(\mathbf{x})
    \enspace .
\end{equation*}
%
Now we need to compute the gradient of $L$ with respect to $\beta$, which is given by
$\nabla L(\beta) = -X^{\top}(y-X\beta)$. Finally, we can replace it in the proximal operator
which yields:

\begin{equation*}
    \beta^{t+1} = \mathcal{T}_{\frac{\lambda}{\mathcal{L}}}\left(
        \beta^{t} + \frac{1}{\mathcal{L}} X^{\top}(y-X\beta^t)
    \right)
\end{equation*}
%
Note that the Lipschitz constant $\mathcal{L}$ for the LASSO is given by $\norm{X^{\top}X}_2$.

\subsection*{FISTA: Fast ISTA}

The idea of FISTA is the same as acceleration à la Nesterov: finding $\beta^{t+1}$ using
an interpolation between $\beta^t$ and $\beta^{t+1}$. It implements a simple heuristic to 
enhance the convergence speed of ISTA (quadratic). In practice, \cite{Bertrand_Massias_Anderson}
showed that FISTA does not accelerate proximal coordinate descent (next section). They
introduce a variant of Anderson extrapolation suitable for coordinate descent, that has 
proved to be very efficient.

\subsection*{Back to CD: proximal coordinate descent}

Let $f : \bbR^p \rightarrow \bbR$ be a convex and differentiable function, and let 
$g_i$ be convex (possibly non-differentiable) function. Let $F$ be:

\begin{equation*}
    F(x) = f(x) + \sum_{i=1}^p g_i(x^{(i)})
\end{equation*}
%
In short, we can separate a function $F$ as a smooth and non-smooth parts. Then, using the same notation 
as previously, the proximal coordinate descent algorithm follows:

\begin{align*}
    &\text{Choose} \enspace i_{k+1} \in \{1, \dots, n\} \\
    &\begin{cases}
        x_{k+1}^{(i)} = \text{prox}_{\gamma_i, g_i}(x_k^{(i)} - \gamma_i \nabla_i f(x_k)) \quad &\text{if} \enspace i = i_{k+1} \\
        x_{k+1}^{(i)} = x_{k}^{(i)} \quad &\text{if} \enspace i \neq i_{k+1} 
    \end{cases}
\end{align*}
%
where $\text{prox}_{\gamma_i, g_i}(y)=\argmin_{x \in \bbR^p} g(x) + \frac{1}{2\gamma_i} \norm{x-y}^2$. 
As we shall see later, a lot of widely-used $g$ have a well-known closed-form solution for proximal operators
(LASSO in particular).
%
In terms of convergence speed, we obtain the same rate as for non-composite objectives.

\subsection*{Active set strategy}

???


\subsection*{Block proximal coordinate descent}

In a multi-output regression setting, blocks of coordinates are updated at each iteration. Let's take Multi-Task
LASSO for instance. 

\begin{equation*}
    \min_{\bfB \in \bbR^{p \times T}}
    \frac{1}{2} \norm{\bf{Y - XB}}^2_\text{F}
    + \sum_{j=1}^p \norm{\bf{B}_{j, :}}_2
\end{equation*}
%
We notice the data-fitting term and a non-smooth regularizing term. When optimizing using
coordinate descent, blocks of coordinates (the rows of $\bf{B}$) are updated at once.

\subsection*{BCD for Multi-Task LASSO}

Let's consider the Multi-task LASSO problem as presented above. We want to derive the proximal operator
for this problem as we did for single-task LASSO.
\\
\\
First, let's derive the proximal operator for $R(\mathbf{B}) = \norm{\mathbf{B}}_{2, 1}$. 
We have shown with ISTA, that the proximal operator for the $\ell_1$ norm is the
soft-thresholding operator. We can reuse this result to derive the proximal operator for
the $\ell_{2,1}$-norm. Indeed, remember that the $\ell_{2,1}$ consists in computing the
$\ell_1$-norm of the row-wise $\ell_2$-norms. The solution is obtained by \textbf{group-soft-thresholding}:

\begin{equation*}
    \text{prox}_{\lambda \norm{\cdot}_2}
    (\mathbf{B}_{j:}) = 
    \left(
        1 - \frac{\lambda}{\norm{\mathbf{B}_{j:}}_2}
    \right)_+
    \mathbf{B}_{j:}
\end{equation*}
%
Now we need to compute the gradient of the data fitting term with respect to $\mathbf{B}_{j:}$
which is given by: 

\begin{equation*}
    \nabla_j f(\mathbf{B}_{j:}) 
    = -\mathbf{X}_{:j}^{\top}
    (\mathbf{Y} - \mathbf{XB})
\end{equation*}
%
This enables us to find the following update rule:

\begin{equation*}
    \mathbf{B}_{j:}^{t+1} = 
    \text{prox}\left(
        \mathbf{B}_{j:}^{t}
        + \frac{1}{\mathcal{L}}
        \mathbf{X}_{:j}^{\top}
        (\mathbf{Y} - \mathbf{X}\mathbf{B}^t)
    \right)
\end{equation*}
%
where $\mathcal{L} = \norm{\mathbf{X}_{:j}^{\top}\mathbf{X}_{:j}}_2$ is the Lipschitz constant.

\newpage
\bibliographystyle{plainnat}
\bibliography{references_all}



\end{document}
